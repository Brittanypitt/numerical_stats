
\documentclass{article}

\usepackage{teaching, array, amsmath}

\DeclareMathOperator{\Prob}{P}
\DeclareMathOperator{\Var}{Var}
\DeclareMathOperator{\Cov}{Cov}
\DeclareMathOperator{\Exp}{E}

\begin{document}

\begin{tdoc}{CHEM 116}{Unit 7, Lecture 3}{Numerical Methods and Statistics}

  \subsection*{Companion Reading}
  \textbf{Langley} Chapter 4


\section{Descriptive Statistics of 2D Data}

Now we will consider data with 2 dimensions. The underlying probability distributions that we assume describes this data now have 2 random variables. We'll return to probability theory to examine this situation and then use statistics to analyze 2D data.

\section{Covariance}
Covariance describes the relationship between two random variables
changing. A positive covariance means the both move in the same
direction. A negative covariance means they move in opposite
directions. The magnitude of the covariance contains information about
both the two random variables' variances and the relationship between
the two. It's defined as:

\begin{equation}
  \Cov(X,Y) = \E\left[(X - \E[X])(Y - \E[Y])\right]
\end{equation}

\begin{equation}
  \Cov(X,Y) = \E[XY] - \E[X]\E[Y]
\end{equation}

{\bf This is for working with random variables, not data!} We can use
some math and such to come up with the following properties of covariance:

\begin{enumerate}

\item $\Cov(X,Y) = 0$, if $X$ and $Y$ are independent\\

\item $\Cov(X,X) = \Var(x)$\\

\item $\Cov(X + Z, Y) = \Cov(X,Y) + \Cov(Z, Y)$\\

\item $\Cov(aX, bY) = ab\Cov(X,Y)$\\

\item $\Cov(X + a, Y + b) = \Cov(X,Y)$\\

\end{enumerate}

\subsection{Example}

Let's say the normal body temperature for a person follows a $\mu = 98.6 ^{\circ}$F and $\sigma=0.5 ^{\circ}$F normal distribution. A fever temperature is exactly $1.1\times T$ ,where $T$ is their body temperature. What's the covariance between fever and body temperature? We are saying that body temperature of a random person follows a random variable and fever is an analytic, deterministic function of body temperature. Now we're asking about their covariance.

\[
\Cov(F, T) = \Cov(1.1\times T, T) = 1.1\times \Cov(T, T) = 1. \Var(T)
\]
\[
\Cov(F,T) = 1.1\times \sigma^2 = 0.275^\circ \textrm{F}^2
\]

After exercise, your temperature is $E = T + I$, where $I$ is an
exponentially distributed random variable with $\lambda =0.25$. What's the covariance between body temperature and post-exercise body temperature? Notice that now we have two \textit{independent} random variables $I$ and $T$, and some summation of the two of them: $E$.

\[
\Cov(E, T) = \Cov(T + I, T) = \Cov(T, T) + \Cov(I, T) = \Var(T)
\]

\[
\Cov(E, T) = (0.5^\circ \textrm{F})^2 = 0.25 ^\circ \textrm{F}.
\]

\section{Sample Covariance}

Just like sample variance and sample mean, there is a sample
covariance. It is connected by the Law of Larger Numbers to covariance. To compute, sample covariance you must have $N$ sets of pairs of data to compute sample
covariance. This is the first time we've been working in pairs by the
way. \textit{The data muse be matched, meaning you are measuring two random
variables in the same sample space}. It might be that your sample space
is a product space; but there must be some pairing in the data.

Examples of invalid `pairs':

\begin{enumerate}

\item How much it snowed today and the total snowfall of the week

\item You have two groups. Group A gets a drug and group B gets a
  placebo. You match each the people up in the group.

\end{enumerate}

Examples of valid `pairs':

\begin{enumerate}

  \item You have people try exercise for 5 weeks and then stop for 5
    weeks. You take their weights after each 5 week period.

    \item You measure a planet's diameter and brightness.

\end{enumerate}


The formula is for sample covariance with your $N$ paired data is:

\begin{equation}
  \sigma_{xy}= \frac{1}{N - 1} \sum_i^N (x - \bar{x})(y - \bar{y})
\end{equation}

Following this notation, sometimes people write sample variance as
$\sigma_{xx}$ instead of $\sigma_x^2$. The reason that $N - 1$ is not
$N - 2$ is that $N$ is the number of {\it pairs} of data points. That
means that we only remove one degree of freedom when we calculate the
mean of $x$ and $y$.

\subsection{Covariance Matrix}

You can write out all covariances/variances in a matrix like so:

\[
\left[\begin{array}{lr}
\sigma_{xx} & \sigma_{xy}\\
\sigma_{yx} & \sigma_{yy}\\
  \end{array}\right]
\]

This is called a covariance matrix. The diagonals are variances and
the off-diagonals are covariances. The covariance can be larger,
depending on the number of random variables, but it's always square.


\section{Sample Correlation}

One of the properties of covariance, and thus sample covariance, is
that $\Cov(X,Y) \leq \sqrt{\Var(x)\Var(y)}$. That means there is a
maximum value. And of course the minimum magnitude is $0$. Thus we can
rescale covariance to get correlation:

\begin{equation}
  r_{xy} = \frac{\sigma_{xy}}{\sigma_x \sigma_y}
\end{equation}

This is the equation for sample correlation.  It runs from $-1$ to $1$
and removes the variance of the two random variables from the
equation.



\end{tdoc}

\end{document}
