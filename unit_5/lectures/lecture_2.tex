\documentclass{article}

\usepackage{teaching, array}

\begin{document}

\begin{tdoc}{CHEM 116}{Unit 5, Lecture 1}{Numerical Methods and Statistics}

  \subsection*{Companion Reading}
  \textbf{Bulmer} Chapter 6

\section{Working With Distributions}

\subsection{Computing the Probability of a Sample}

You are on Facebook. You see a post. The probability you like a post
is 0.10. What kind of distribution is this? Bernoulli because there
are two outcomes and fixed probability for each. [Draw a Picture]

You scroll down Facebook and decide to stop once you like a single
post. Again you like a post with probability 0.1. What is the
probability you liked a single post and stopped after seeing 10
posts? This is a geometric distribution because we stop as soon as we
succeed and the number of trials is unbounded. [Draw a Picture]
\[
P(n=10) = (1 - 0.1)^{10 - 1}(0.1) = 0.039
\]

You no longer stop after liking a post, you just scroll through a
fixed number. What is the probability you liked 3 posts after seeing
10 posts? This is a binomial distribution because it has a fixed
number of trials and the order does not matter. [Draw a Picture]
\[
P(n=3) = {10 \choose 3} (1 - 0.1)^{10 -3 }(0.1)^{3} = 0.057
\]

You've become very cynical and only like 2\% of posts now. You scroll
through 250 posts. What is the probability you liked 4 posts? This is
a Poisson distribution because it is the same as set-up as a binomial
but the probability is low and the trial number is high. First, to
get the parameter for the Poisson, $\mu$, we must compute it. It's
the expected number of successes $\mu = Np = 250 \times 0.02 = 5.0$
[Draw a Picture]
\[
P(n=4) = \frac{e^{-5}5^4}{4!} = 0.175
\]

 \subsection{Computing the Probability of an Interval}

 All of these examples have small probabilities because we're looking
 at single points. What about intervals.

Return to the geometric example. What's the probability that you liked
a post and stopped scrolling in less than 10 posts? [Draw a picture]

\[
P(n = 1) + P(n = 2) + ... = \sum_{i=1}^{9} P(n = i)
\]
Using Python...
\[
P(n < 10) = 0.724
\]

What about the opposite? What's the probability it took more than or
10 posts?
\[
P(n >= 10) = 1 - P(n < 10) = 0.276
\]
[Draw a picture]. An example of the NOT rule.

We can use the same equation, $\sum_{i=1}^N P(i)$, to compute these
intervals with discrete probability mass functions.

\subsection{Computing the Probability of an Interval for Continuous Distributions}

Let's assume Facebook posts arrive according to an exponential
distribution. This makes sense because time is continuous, cannot be
negative, and exponential distributions describe the time between
random events. If on average 10 posts occur per hour, $\lambda$ the
exponential parameter is $1/6$ inverse minutes. What is the probability of a
new post within the 5 minutes? [Draw a picture]
\[
\Pr(t < 5) = \int_0^{5} \lambda e^{-\lambda t}\,dt = -\frac{\lambda}{\lambda}\left.e^{-\lambda t}\right]_0^{5} = 0.63
\]

What is the probability a new post arrives after 2 minutes but before 10 minutes?
\[
\Pr(2 < t < 10) = \int_{2}^{10} \lambda e^{-\lambda t}\,dt  = 0.53
\]
[Draw a picture]

What is the probability the new post comes after 4 minutes?
\[
\Pr(t > 4) = \int_4^{\infty}  \lambda e^{-\lambda t}\,dt = \left.e^{-\lambda t}\right]_4^{\infty} = 0.51
\]
[Draw a picture]

You decide to stop playing on the Internet so much, but you only like
going outside if the temperature is greater than 65 degrees and less
then 80 degrees. Assume the temperature has a mean of 45 degrees in
Rochester and the standard deviation is 15 degrees. Assume you measure
the temperature once per day. What's the probability you'll go outside?
\[
\Pr(65 < T < 80) = \int_{65}^{80} \frac{1}{\sigma\sqrt{2\pi}}e^{\frac{-(x - \mu)^2}{2\sigma ^2}} = 0.15
\]
Note, the above equation must be evaluated numerically. [Draw a
  picture]

\subsection{Finding Prediction Intervals}

Let's return to the geometric distribution. What if we're interested
in solving this equation? [Draw a Picture]
\[
P(n < x) = 0.9
\]
What is the number of trials for which there is a 90\% probability we
succeeded? This called a \emph{Prediction Interval}. This generally
must be solved with Python. The answer is 15. Compare that with the
expected value, which is 10. The answer to the question: ``How long
will it take to find a post you like?'' is about 10. A prediction
interval allows us to give a worst-case value, where we are wrong
about it 10\% of the time. We state it as ``With a prediction level of
10\%, the number of posts I will see before liking one is 15''.


\end{tdoc}

\end{document}
