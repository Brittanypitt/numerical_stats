\documentclass{article}

\usepackage{teaching, array}

\begin{document}

\begin{tdoc}{CHEM 116}{Unit 4, Lecture 2}{Numerical Methods and Statistics}

\section{Expected Values}

The expected value of a samlpe $x$ is
\begin{equation}
\pexp{X} = \sum_\mathcal{Q} P(X=x)x
\end{equation}

An expected value is analagous to a mean, except you don't add data
points. You add the elements in your sample space, weighted by their
probability. Be careful not to be confused: the expected value is not
the most likely outcome. The most likely outcome is the $x$ which
maximizes $P(x)$, written as:

\begin{equation}
\argmax_x P(x)
\end{equation}

\subsection{Die Roll Example}

Consider a fair die example, where $d$ is an element in our sample
space. Our sample space is one through and the probability of a sample
occuring is $1/6$. 

\[
\pexp{D} = \frac{1}{6}\times{}1 + \frac{1}{6}\times{}2 + \frac{1}{6}\times{}3 \ldots
\]

\[
\pexp{D} = \frac{1}{6}\times 21 = 3.5
\]

\subsection{Unfair Die Example}
Now consider our unfair die, where
\[
P(D = d) = \frac{d}{21}
\]

so for example, the probability of rolling a 4 is $4/21$.

\[
\pexp{D} = \sum_{1,\ldots,6} \frac{d}{21} \times d = \frac{91}{21}
\]

\section{Expected Value of a Random Variable}

To calculate the expected value of a random variable $S$, use:

\begin{equation}
\pexp{S} = \sum_\mathcal{Q} P(X=x)S(x)
\end{equation}

Remember that the random variable $S$ is defined by some $S(x)$
function, whose input is an element in the sample space and has an
output of some real number.


\subsection{Expected Value for Sum of 2 Dice}

Consider the product sample space of two dice: $Q_2 = Q \otimes Q$,
where $Q$ is $1,\ldots,6$. Example of elements of this sample space
are rolling a 2 for the first die and a 4 for the second die: $(2,
4)$. There are 36 elements in the sample space and we'll take the
probability of each to be equivalent. This gives:

\[
P(X = x) = P(X = (x_1, x_2)) = \frac{1}{36}
\]

Define $S$ as our random variable representing the sum of the two dice.

\[
S(x) = S(\, (x_1, x_2)\, ) = x_1 + x_2
\]

Now we can plug these into our expected value equation:

\[
\pexp{S} = \sum_{Q} P(X=x)S(x) = \sum_{(x_1,x_2) \in Q} \frac{1}{36} S( (x_1, x_2) )
\]

\[
= \frac{1}{36}\times (1 + 1) + \frac{1}{36} \times (1 + 2) + \frac{1}{36} \times (1 + 3) + \ldots + \frac{1}{36} \times (6 + 6)
\]

To calculate the answer, we need to sum all 36 possible dice rolls
together. That turns out to be 252. So we have:

\[
\pexp{S} = \frac{252}{36} = 7
\]

\subsection{Expected Value for Sum of 2 Dice - Alternate}

Another way to solve this problem is to DEFINE our sample space to be
the possible sums. Then $Q = \{2, 3, \ldots, 12\}$. Our correponding
probability is going to be the number of combinations to roll the sum
divided by some normalizing constant. For example, the probability of
rolling a 6 is 5 / Z, where $Z$ is some constant to make sure our
probabilities sum to 1. The 5 comes from the number of ways to roll a
6: $(1, 5), (2, 4), (3, 3), (4, 2), (5, 1)$. Now we to find $Z$:
\[
Z = \underbrace{1}_{\textrm{combinations to roll a 2}} + \underbrace{2}_{\textrm{combinations to roll a 3}},\, \ldots\,, \underbrace{1}_{\textrm{combinations to roll a 12}}
\]
\[
Z = \underbrace{1}_2 + \underbrace{2}_3 + \underbrace{3}_4 + \underbrace{4}_5 + \underbrace{5}_6 + \underbrace{6}_7 + \underbrace{5}_8 + \underbrace{4}_9 + \underbrace{3}_{10} + \underbrace{2}_{11} + \underbrace{1}_{12} = 36
\]

where the underbraces indicate which dice roll sum each term corresponds
to. Doing this work also reveals an equation for the probability:

\[
P(S = s) = \frac{6 - |s - 7|}{36}
\]

Finally, we are ready to utilize the expected value equation:

\[
\pexp{S} = \sum_\mathcal{Q} P(S=s)s = \sum_{2,\ldots,12} \frac{6 - |s - 7|}{36}\times s 
\]
\[
= \frac{1 \times 2}{36} + \frac{2 \times 3}{36} + \frac{3 \times 4}{36} + \ldots + \frac{1 \times 12}{36} 
\]
\[
= \frac{252}{36} = 7
\]

Thus we've arrived at the same answer. 

\subsection{Continuous Expected Value}

\begin{equation}
\pexp{X} = \int_\mathcal{X}\,xp(x)\,dx
\end{equation}


\subsection{Continuous Example}
\[
p'(x) \propto x,\:  Q = [0,5]
\]

First, we must normalize it:
\[
\int_0^5 \,x\,dx = \frac{25}{2}
\]
\[
p(x) = \frac{2x}{25}
\]

\[
\int_0^5 x\times\frac{2x}{25} = \frac{2\times 125}{25\times 3} = \frac{10}{3}
\]

\subsection{Conditional Expectation Value}
\begin{equation}
\Econd{X}{Y=y} = \sum_\mathcal{X} \Pcond{X=x}{Y=y}x
\end{equation}

\subsection{Die Roll Example}

Our random variable $X$ is the observation, and $Y = 0$ if the observation is odd and $Y = 1$ if the observation is even.

\[
\Econd{X}{Y=0} = \frac{1}{3}\times{}1 + \frac{1}{3}\times{}3 + \frac{1}{3}\times{}5 = 3
\]

Expected value gives the center of a random variable or probability
distribution.

\section{Variance}

Variance gives the average deviation from that
center. To make sure variation above and below the center contributes,
it is the expected squared distance from the expected value:
\begin{equation}
\var(X) = \pexp{(\pexp{X} - X)^2} = \pexp{X^2} - \pexp{X}^2
\end{equation}

\subsection{Die Roll Example}
We already know $\pexp{X} = 21/6 =  3.5$. To find $\pexp{D^2}$, we can use our
random variable expectation equation. Take $F$ to be our random
variable deinfed by $F(x) = x^2$:
\[
\pexp{D^2} = \pexp{F} = \sum_Q P(x)x^2 = 
\]
\[
= \frac{1}{6}\times{}1^2 + \frac{1}{6}\times{}2^2 + \frac{1}{6}\times{}3^2 \ldots
\]
\[
 = \frac{91}{6}
\]
\[
\var(D) = \frac{91}{6} - \frac{21^2}{6^2} = \frac{35}{12} = 2.92
\]

\end{tdoc}

\end{document}

