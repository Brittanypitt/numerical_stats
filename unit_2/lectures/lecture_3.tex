\documentclass{article}

\usepackage{teaching, array}

\begin{document}

\begin{tdoc}{CHEM 116}{Unit 2, Lecture 3}{Numerical Methods and Statistics}

  \section{Equations with Random Variables}

  \subsection*{Companion Reading}
  \textbf{Bulmer} Chapter 3

\subsection{Joint Probability Distribution}
We will indicate the probability that two rvs, $X$ and $Y$, adopt a
particular value $x$ and $y$ \emph{simultaneously} as $\Pr(x,y)$. This
is called a joint probability distribution. Joints indicate
simultaneous occurance, unlike our {\bf AND} from last lecture.

To treat successive observations, we simply rearrange our current
definitions. Take flipping a coin. We redefine our sample space to be
the product space of trials flip 1 and flip 2, so $(H,H)$, $(H,T)$,
$(T,H)$ and $(T,T)$ where $H=$heads, $T=$tails. Next, we say $X$ is the
observation of trial 1 and $Y$ is the observation of trial 2.

The continuous analogue is $p(x,y)$, which is not meaningful unless
integrated over an area. Example: observing particle in fixed area.

\subsection{Marginal Probability Distribution}
The marginal probability distribution function is defined as
\begin{equation}
\Pr(x) = \sum_y P(x,y)
\end{equation}
The marginal means the probability of $X=x, Y=y_0$ \textbf{OR} $X=x, Y=y_1$ \textbf{OR}
$X=x, Y=y_2$, etc. For example, if $X$ is the weather and $Y$ is the
day of the week, it is the probability of the weather being a particular value (e.g., nice) `averaged' over
all possible weekdays.

The marginal allows us to remove a rv or sample space dimension. That
process is called {\bf marginalizing} and the resulting $\Pr(x)$ is
called the marginal. Marginalization is especially important if the two pieces of
the joint are not independent.

\subsection{Conditional Probability Distribution}
A conditional probability distribution allows us to fix one sample,
event, or rv in order to calculate another. It is written as
$\Pr(X=x|Y=y)$. For example, what is the probability of having the flu
given I'm showing cold/flu symptomps. Conditionals are generally much
easier specify, understand and measure than joints or marginals.

\begin{itemize}
\item The probability I visited node C given that I started in node A and ended in node B
\item The probability a test shows I have influenza given that I do not have influenza (false negative)
\item The probability that the sum of two dice is 7 given that one die shows 4 when rolling two dice.

\end{itemize}

The definition of a conditional probability distribution is
\begin{equation}
\Pr(x|y) = \frac{\Pr(x,y)}{\Pr(y)}
\end{equation}

A CPD is a full-fledged PMF, so $\sum_\mathcal{X} \Pr(x|y) = 1$ due to
normalization, sometimes called the law of total probability.  If we
forget what goes in the denominator on the right-hand-side we can
quickly see that $\sum_\mathcal{X} \Pr(x,y) / \Pr(y) = 1$ whereas
$\sum_\mathcal{X} \Pr(x,y) / \Pr(x) \neq 1$.

The definition is the same for continuous distributions.

This leads to an alternative way to marginalize:
\[
\sum_\mathcal{Y} \Pr(x|y) \Pr(y) = \sum_\mathcal{Y} \Pr(x,y) = \Pr(x)
\]


\subsection{Viewing Conditionals as Sample Space Reduction}

Consider guessing binary numbers at random between 0 and 7:\vspace{0.25cm}

\begin{tabular}{l}
000\\
001\\
010\\
011\\
100\\
101\\
110\\
111\\
\end{tabular}\vspace{0.25cm}

The probability of sampling 4 (100),
\[
\Pr(x = 100) = \frac{1}{8}
\]
Now, consider the rv $Y$, the number of non-zero bits. What is
\[
\Pr(x = 100 | Y = 1)
\]
We could rewrite this in terms of the joint and marginal, as
\[
\Pr(x = 100 | Y = 1) = \frac{(x = 100, Y = 1)}{Y = 1} = \frac{1 / 8}{3 / 8} = \frac{1}{3}
\]
Or we could recognize that the condition of $Y = 1$ reduces the sample
space to 3, because there are only 3 samples that are consistent with
$Y = 1$. Thus, the probability of $x = 100$ is $1/3$, since $x = 100$
has a single permutation and $Q_c$, the conditional sample space is
$3$.

\section{Tricky Concepts}

\begin{description}

\item[Product Spaces] A product space is for joining two possibly
  dependent sample spaces. It can also be used to join sequential trials.

\item[Event vs Sample on Product Spaces] Things which were samples on
  the components of a product space are now events due to permutations

\item[Random Variables] They assign a numerical value at each sample
  in a sample space, but we typically care about the probability of
  those numerical values (PMF). So $X$ goes from sample to number and
  $\Pr(x)$ goes from number to probability.

\item[Continuous PDF] A pdf is a tool for computing things, not
  something meaningful by itself.

\item[Marginal Probability Distribution] A marginal `integrates/sums'
  out other samples/random variables/events we are not interested in.

\item[Joint vs Conditional] People almost never think in terms of
  joints. Conditionals are usually easier to think about, specify, and
  be a way to attack problems.

\end{description}

 \section{Working with Marginals, Condtionals, and Joints}

\subsubsection{Seasons Example}

The sample space is a product space of the seasons $T$ (Winter (0), Spring
(1), Summer (2), Fall (3) ) and $W$ if the weather is nice or not (N=nice,
S=not nice). We know that
\[
\begin{array}{lr}
\Pr(W=N|T=0) = 0 & \Pr(W=N|T=1) = 0.4\\
\Pr(W=N|T=2) = 0.8 & \Pr(W=N|T=3) = 0.7\\
\end{array}
\]
\[
P(T=t) = \frac{1}{4}
\]

What is the probability that the weather is not nice? Use
marginalization of conditional:

\[
\Pr(W=S) = \sum_\mathcal{T} \Pr(W=S | T) P(T) = 0.25(1 - 0) + 0.25(1 - 0.4) + 0.25(1 - 0.8) + 0.25(1 - 0.7)
\]
\[
\Pr(W=S) = 0.525
\]

What is the probability that it is fall given that it is nice? Start with definition of conditional :
\[
\Pr(T=3|W=N) = \frac{\Pr(T=3, W=N)}{\Pr(W=N)}
\]
We know that $\Pr(T=3, W=N) = \Pr(W=N|T=3)\Pr(T=3)$ due to definition of conditional:
\[
\Pr(T=3|W=N) = \frac{\Pr(W=N|T=3)\Pr(T=3)}{\Pr(W=N)}
\]
Finally, we can use $\Pr(W=N) = 1 - \Pr(W=S)$, the {\bf NOT} rule:
\[
\Pr(T=3|W=N) = \frac{0.7\times 0.25}{1 - 0.525} = 0.368
\]

\subsection{Bayes' Theorem}

We derived a well-known equation in that example called Bayes' Theorem:

\begin{equation}
\Pr(A|B) = \frac{\Pr(B|A)\Pr(A)}{\Pr(B)}
\end{equation}

This is useful to swap the order of conditionals

\subsection{Independence}
Finally, we are ready to define independence. If the random variables
$X$ and $Y$ are independent, then
\begin{equation}
\Pr(x|y) = \Pr(x)
\end{equation}
This implies via Bayes' Theorem
\begin{equation}
\Pr(y|x) = \Pr(y)
\end{equation}
And also implies via CPF definition
\begin{equation}
\Pr(x,y) = \Pr(x)\Pr(y)
\end{equation}
which was our {\bf AND} rule from last lecture.

\vspace{0.2cm}
In our weather example, is the season and weather independent?
\[
\Pr(W=N|T=0) \neq P(W=N|T=1)
\]
so no.


\subsection{Compound Conditionals}
When writing conditionals, this is a common short-hand:
\[
\Pr (x = 2\, | \,Y = A, Z = 0)
\]
for
\[
\Pr ([x = 2]\,|\, [Y = A, Z = 0])
\]
The conditional is always evaluated last, for example:
\[
\Pr(x = 2, Y = A\, | \,Z = 0)
\]
is also possible and means the probability of the joint $(x=2, Y=A)$
given that $Z = 0$



\subsection{Conditional Independence}
$X_0$ and $X_1$ are conditionally independent given $Z$ if
\begin{equation}
\Pr(X_0 | Z, X_1) = \Pr(X_0 | Z)
\end{equation}

which is equivalent to

\begin{equation}
\Pr(X_0, X_1 | Z) = \Pr(X_0 | Z)\Pr(X_1 | Z)
\end{equation}


To use conditional independence, you must condition on $Z$. For
example, if I want to calculate $\Pr(X_0, X_1)$, I'll need to
condition it. Marginalizing a conditional is one way to get that
quantity, using a compound conditional:
\[
\Pr(X_0, X_1) = \sum_\mathcal{Z}\Pr(X_0, X_1 | Z) \Pr(Z)
\]
Now it is in a form where the conditional independence applies:
\[
\Pr(X_0, X_1) = \sum_\mathcal{Z}\Pr(X_0 | Z) \Pr(X_1 | Z)
\]
This is common for sequential trials, where the trials are independent
when conditioned on some underlying property, but dependent if we do
not know the property. For example, let's say I have two dice, one
that is fair and one that follows the biased die model we saw in
class. Let's further assume that $P(D = 0) = 0.1$, where $D$ indicates
the chosen die.
\[
\Pr(X=x | D = 0) = \frac{1}{6}
\]
\[
\Pr(X=x | D = 1) = \frac{x}{21}
\]

If I know which die I'm rolling, then every roll is independent
as expected:
\[
\Pr (X_0 = 6, X_1 = 1 | D) = \Pr (X_0 = 6 | D) \Pr (X_1 = 1 | D)
\]
however, consider
\[
\Pr (X_1 = 1 | X_0 = 6) = \frac{\Pr(X_1 = 1, X_0 = 6)}{\Pr (X_0 = 6)}
\]
As above, we'll try to condition it to exploit conditional
independence.
\[
\Pr(X_1 = 1, X_0 = 6) = \Pr(X_1 = 1 | D = 0) \Pr(X_0 = 6 | D = 0) \Pr (D = 0) +
\]
\[
\Pr(X_1 = 1 | D = 1) \Pr(X_0 = 6 | D = 1) \Pr (D = 1)
\]
\[
\Pr(X_1 = 1, X_0 = 6) = \frac{1}{6}\frac{1}{6}\frac{1}{10} + \frac{1}{21}\frac{6}{21}\frac{9}{10} = 0.0150
\]
\[
\Pr(X_0 = 6) = \sum_{X_1} \Pr(X_1, X_0 = 6) = \frac{1}{6}\frac{1}{10}\underbrace{\sum_{x} \frac{1}{6}}_{=1} + \frac{6}{21}\frac{9}{10} \underbrace{\sum_{x} \frac{x}{21}}_{=1}
\]
\[
\Pr(X_0 = 6)  = 0.274
\]
\[
\Pr (X_1 = 1 | X_0 = 6) = \frac{0.0150}{0.274} = 0.0549
\]
Finally, to show they are not independent:
\[
\Pr(X_1 = 1) = \sum_{X_1} \Pr(X_1 = 1, X_0) = \frac{1}{6}\frac{1}{10}\underbrace{\sum_{x} \frac{1}{6}}_{=1} + \frac{1}{21}\frac{9}{10} \underbrace{\sum_{x} \frac{x}{21}}_{=1} = 0.0595
\]

The intuition here is that the marginal of $X_1 = 1$ considers both
die according to the die marginals. However, knowing that $X_0 = 6$
shifts it so that the biased die is more likely. This can be seen with
Bayes theorem to:
\[
\Pr(D = 1 | X_0 = 6) \neq \Pr (D=1)
\]
and thus $\Pr (X_1 | X_0 = 6)$ changes from $\Pr(X_1)$.

\subsection{Table of Equations}

  \begin{tabular}{lr}
    $P(x) = \sum_y P(x,y)$ & Definition of Marginal\vspace{0.15cm}\\
    $P(x|y) = \frac{P(x,y)}{P(y)}$ & Definition of Conditional\vspace{0.15cm}\\
    $P(x,y) = P(x|y)P(y)$ & Definition of Conditional\vspace{0.15cm}\\
    $\sum_x P(x|y) = 1$ & Law of Total Probability/Normalization\vspace{0.15cm}\\
    $\sum_y P(x|y)P(y) = P(x)$ & Marginilzation of Conditional\vspace{0.15cm}\\
    $P(x|y) = \frac{P(y|x)P(x)}{P(y)}$ & Bayes' Theorem\vspace{0.15cm}\\
    $P(x,y) = P(x)P(y)$ & Definition of Independence\vspace{0.15cm}\\
    $P(x | y) = P(x)$ & Independence Property (x,y independent)\vspace{0.15cm}\\
    $P(x,y | z) = P(x|z)P(y|z)$ & Conditional Independence\vspace{0.15cm}\\
  \end{tabular}

\end{tdoc}

\end{document}
